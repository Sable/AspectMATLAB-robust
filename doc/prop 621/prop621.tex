% Sample file for COMP 621 Project Propsal

\documentclass{comp621}

\usepackage{epsfig} % so you can include .eps figures
\usepackage{alltt}  % so you can include program text 
% add other packages you might use here
\usepackage{enumitem}
\usepackage{calc}
\setlist{nosep}

% Fill in your title, project number and authors
\title{Optimizing a robust AspectMATLAB front-end}
\ReportNumber{2016-09} % replace  yyyy with a year and xx with your project #
\author{Hongji Chen}
\date{November 1, 2016}

% Main body of the Document begins here
\begin{document}

\MakeTitlePage

\tableofcontents

% \listoffigures comment out if you have no figures
% \listoftables  comment out if you have no tables

\clearpage

% Replace each subsection below with the appropriate contents.
% Feel free to add more (sub)sections if you want. Write clearly
% and concisely. 

\section{Introduction and Motivation}

MATLAB is a widely used dynamic scientific programming language, and
AspectMATLAB aims to introduce Aspect-oriented programming (AOP) into MATLAB. 
AOP allow users to match program points with pointcuts (patterns in
AspectMATLAB) and apply  user-defined pointcut advice (actions in
AspectMATLAB).  One major possible application of AspectMATLAB is programming
profiling. Although current MATLAB has integrated with built-in profile
features but is relative limited and difficult to customize. Like AspectJ and
many other AOP implementation AspectMALAB provide user to capture program
points using patterns and retrieve runtime information using content exposure
selectors. But unlike most of AOP implementation, AspectMATLAB allow users to
match program points not only by type and scope information but also by shape
information. However, the dynamic features that are utilized by MATLAB pose
several challenges for source code transformation and weaving. In order to
match the user defined patterns, AspectMATLAB needs to insert runtime checks to
collect the type and shape information about the program points, and such
runtime checks are proven to be costly. This project is aimed to first create a
robust transforming framework for AspectMATLAB and then implemented several
static analysis to improve the performance of the weaved MATLAB source code.

\section{Background and Related Work}

The project is built on the success of Aslam's AspectMatlab
compiler\cite{AspectMATLABpaper} , Bodzay's AspectMatlab++ compiler
\cite{AspectMATLABPPpaper} , and summer research on "Creating a robust
front-end for AspectMATLAB". 

\subsection*{Aslam's AspectMatlab Compiler\cite{ToheedMasterThesis}}
This is the first version of AspectMatlab compiler, it lays the  foundation for
later versions of AspectMatlab compiler. In the paper,  eight patterns are
purposed. 

\begin{description}[labelwidth =\widthof{\bfseries Execution}]
\item[Call] captures all call to function or scripts
\item[Execution] captures the execution of function bodies
\item[Get] captures array accesses
\item[Set] captures array sets
\item[Loop] captures execution of all loops
\item[LoopHead] captures the header of the loop
\item[LoopBody] captures the body of the loop
\item[Within] restricts the scope of matching
\end{description}

For each join point, AspectMatlab offers three type of weaving,
\textbf{before}, \textbf{after} and \textbf{around}. The transforming strategy
used by this version of AsepctMatlab compiler is relatively straight forward,
and in most of the cases, optimization can be applied to improve the execution
performance.

\subsection*{Bodzay's AspectMatlab++ compiler\cite{AndrewMasterThesis}}
AspectMatlab++ compiler is built on the success of the original AspectMatlab
compiler, and extended with \textbf{annotation}, \textbf{type} and
\textbf{dimension} pattern. The \textbf{annotation} pattern allow user to
define special formatted comment, and insert action calls at comment position.
The \textbf{type} and \textbf{dimension} pattern behave similar with
\textbf{within} pattern, but instead of restricting the matching by scope, it
check the type and shape information of the join point. Despite the enhancement
in the patterns, the AspectMatlab++ compiler utilizes the function inlining
technique. Unlike the original AspectMatlab compiler which insert a function
call at every join point, the AspectMatlab++ compiler will attempt to insert
the action directly into the weaved source code. The experiment result shows
that the function inlining can effectively boost the performance of the weaved
code.

\subsection*{Summer research}
However, in both existed version of AspectMatlab compiler, the pattern does not
checked before weaved. Thus, it is possible that it accept some semantically
invalid pattern, which could lead to unpredictable result for the generated
weaved code. Thus, a more clear grammar and a set of semantic validation rules
has been developed, such that the new compiler will reject the semantically
invalid pattern before proceeding further in the compilation stages.

The transform strategy used in the existed version of AspectMatlab and
AspectMatlab++ compiler has been out dated, and some weaved code cannot be
correctly interpreted in the latest version of MATLAB (2016a). Thus, another
part of the summer research is to develop a new set of transforming rules that
are compatible with the latest version of MATLAB. But currently, the
transforming strategy is straight forward and have no optimization applied.
Thus, this project is aim to optimize the weaved source code generated by the
new AspectMATLAB compiler.

\section{Specific Problem Statement}

In order to correctly transform the source code for action weaving with minimum
development complexity, the transformer is designed to be straight forward, and
it opens the opportunity for further optimization. The project is aimed to
implement the following optimization of the generated weaved code:
\begin{enumerate}
    \item data-flow analysis discussed in the lecture
    \item a analysis to extract and minimize the number of dynamic checks
          required
    \item a procedure to determine if a action can be directly inlined into the
          weaved source code
\end{enumerate}

\section{Solution Strategy}
One approach to extract and minimize the number of dynamic checks is to design
an analysis similar with common expression analysis that eliminate redundant
dynamic checks. Another approach is to purpose a type and shape approximation
analysis on AST, and then the compiler can statically decide whether a program
point is matched by a pattern.

In order to determine if an action can be directly inlined into the weaved
source code, the first step is to collects a set of conditions where the action
must satisfied, and then purpose a set of analysis procedure to verify the
conditions.

\section{Experimental Framework}

The new AsepctMATLAB is built on new McLab framework that modified during the
summe. Also a new parser\cite{MclabParser}, developed by Samuel Suffos
during the summer will also be used in the new AspectMATLAB compiler, this will
solve most of the parsing errors in the existed AspectMATLAB compiler. The
testing and benchmarking will be performed on the latest MATLAB environment
(2016b), as the existed AspectMATLAB compiler is no longer compatible with the
new version of MATLAB. Further, the new AspectMATLAB compiler will use Wu-wei
benchmark toolkits for benchmarking and profiling.

\section{Schedule of Activities}

The first part of the project is to finish the implementation of the
transforming framework. The design of the framework and transforming rules have
been developed over the summer, and hopefully, the transforming framework can
be finished shortly. The second part of the project is to implement several
analysis discussed within the lectures, including constant propagation, and
common expression analysis. The third part of the project to modify the common
expression analysis such that it can correctly extract the dynamic query
expression inserted by the transform framework. The last part of the project is
to develop the action inlining strategy. The first part and the second part is
expected to be finished within short period of time, and the project is focus
on the third and the fourth part.

\section{Expected Results and Evaluation Method}

The new AspectMATLAB compiler should be more robust than the existed one, and
it can handle all the patterns correctly. Also, the new AspectMATLAB should
perform relatively better than the existed version. One method to evaluate the
result is to insert empty action into the source code, and measure the
execution time with the benchmark toolkits. This will make sure that the change
in execution time directly reflects the performance of the compiler.

% Comment out the next 2 command lines if you have no bibliography:
% However, your should have one!
\bibliographystyle{plain}
% You can add more entries into example.bib, or add your own .bib file
% and add it to the bibliography list below
% \bibliography{example}
\bibliography{reference}

\end{document}
