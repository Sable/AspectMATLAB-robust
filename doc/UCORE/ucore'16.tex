\documentclass[preprint]{sig-alternate-05-2015}

\title{Creating a Robust AspectMATLAB compiler}
\doi{}
\isbn{}


\numberofauthors{2}
\author{
\alignauthor
Hongji Chen\\
			\affaddr{McGill University}			\\
			\affaddr{Montreal, QC, Canada}		\\
			\email{hongji.chen@mail.mcgill.ca}
\alignauthor
Laurie Hendren (advisor)\\
			\affaddr{McGill University}			\\
			\affaddr{Montreal, QC, Canada}		\\
			\email{hendren@cs.mcgill.ca}
}

\begin{document}
\maketitle

\section{Problem and Motivation}

%LJH add a citation for MATLAB - fixed
MATLAB \cite{MATLABWeb} is a dynamic programming language that widely
used in scientific computation.  Aspect oriented programming (AOP) allows
programmers to separate out concerns by defining ``pointcuts" which identify
certain program events such as writing to a variable,  or entering a function,
and then associate with those pointcuts advice, or actions, which will
execute at those points.  This is accomplished by the compiler weaving
the advice codes into the appropriate program locations that match the
pointcuts.

This project aims to build on previous work to create a robust and
semantically well-defined version of AspectMATLAB.  In addition, more
powerful patterns and matching strategies have been developed and
implemented,  extending the functionality of AspectMATLAB.  Having a
robust and well-defined AspectMATLAB will enable scientists to use
aspects to add functionality to their base MATLAB programs. Example uses
include sparse matrix detection, dynamic type checking, custom profiling, and
much more.


\section{Background and Related Work}

%LJH add citations for AspectJ and Jastadd - fixed
AspectJ \cite{AspectJ}, an AOP implementation for Java, has had a huge
influence in the Java world, and Jastadd \cite{JastAdd} framework
utilizes the idea of AOP to shorten the development period for compiler by
generating AST structure from aspects. 

Our research project is based on the success of Toheed Aslam's
AspectMATLAB compiler \cite{AspectMATLABpaper} and Andrew Bodzay's
AspectMATLAB++ compiler \cite{AspectMATLABPPpaper}. The existing
AspectMATLAB compiler allows programmers to define patterns(``pointcuts")
and actions(``advice"). Along with variable access, function call and
execution, which are present in most AOP languages, the existing
AspectMATLAB compiler introduces loop patterns to capture loop statement
\cite[p.15-17]{ToheedMasterThesis}, and shape matching and attribute matching,
which are both important in scientific computing
\cite[p.14-15]{AndrewMasterThesis}.


\section{Approach and Uniqueness}

The existing compiler demonstrated that the idea of AspectMATLAB was
interesting and worthwhile,  but it has many weaknesses including:
incomplete source code parsing, undefined and unimplemented pattern
validation, an incomplete transforming strategy, and limitations on
argument parsing.  The key motivation of this project was to identify
all of the weaknesses and to improve and extend the language definition
and the compiler.  As a result of this project, the new AspectMATLAB
language definition and the compiler is much more complete and robust.

\section{Results and Contributions}

The project has resulted in a new implementation of the AspectMATLAB
compiler which improves over the previous ones in the following ways:

\begin{description}
\item[Use of more Robust front-end:]
The previous AspectMATLAB implementations used an incomplete front-end,
and hence could not parse many MATLAB Programs. This project integrates a much
more robust front-end using Samuel Suffos's new MATLAB parser
\cite{MclabParser}. 

\item[Clear Grammar:] The existing AspectMATLAB compiler provides 11
different types of patterns, but some of the patterns have confusing
syntax and/or semantics. We also noticed that some of the patterns are
usually used together, such as variable get pattern usually forms a
compound pattern with dimension pattern, and type pattern. Thus, we
developed a new set of pattern grammar rules to remove the ambiguous and
redundant patterns, while also making it backward compatible with the
existing grammar.

\item[Semantic Validation:] Limited semantic validation is applied on
the patterns in the existed AspectMATLAB compiler. Semantic invalidation
will make the source code hard to read and even lead to unpredictable
results during code transformation. Thus we have developed a
simplification and analysis process for patterns. This includes pattern
type analysis, pattern modification validation, pattern syntax
validation. The current version of AspectMATLAB will raise errors and
warnings during compilation.

\item[More Robust Transforming Strategy:] We also improved the
transforming and weaving strategy for AspectMATLAB. The existing
AspectMATLAB uses a simple transformation strategy on cell indexing,
variable set and comma separate list handling. The original
transformation resulted in improper resolving of assigned values and cannot
correctly identify the spanned comma separate list. We have implemented a
new transformation framework and a sound transforming strategy, making
the compiler more robust.

\item[Extended Argument Matching:] In the existing AspectMATLAB,
programmers are only allowed to match the function call and execution
using the number of arguments. The new AspectMATLAB takes this a step
further. We have developed a new matching strategy for argument list
which allows the programmer to specify both type and shape information
for each argument, as well as using wildcards to make a general match.
This general matcher is implemented as a generalized DFA.
\end{description}

This project has identified many weaknesses in the previous AspectMATLAB
compilers, and has developed a new version of AspectMatlab, which builds
upon a more robust front-end, addresses many shortcomings in the
previous AspectMATLAB versions, and provides extensions to the
AspectMATLAB language.    

The new AspectMATLAB compiler can be used in different kinds of MATLAB
optimization, such as sparse matrix tracing, subroutine profiling, loop
unrolling and dynamic call graph generation. It can also be used to
generate safer  MATLAB code, by using aspects for argument type and
shape checking. Another important usage is to provide analysis and
profiling features using predefined aspects for other MATLAB toolkits,
for example, McWeb, a web-based interface for MATLAB analysis, profiling
and source-to-source compilation.

\section*{Acknowledgements}
This work was done as a summer research project, supported by a SURA and
NSERC. The authors would like to thank the donors who provide the
support for the SURA awards. We would also like to thank Samuel
Suffos for his help in integrating his new front-end with AspectMATLAB.

\bibliographystyle{plainurl}
\bibliography{reference}

\end{document}
